\documentclass[12pt]{article}

% --- Página y tipografía ---
\usepackage[letterpaper,margin=2.5cm]{geometry}
\usepackage[T1]{fontenc}
\usepackage[utf8]{inputenc} % si compilas con pdfLaTeX
\usepackage{lmodern}
\usepackage{microtype}

% --- Imágenes y color ---
\usepackage{graphicx}
\usepackage{xcolor}

% --- Control fino de espacios ---
\usepackage{setspace}
\setlength{\parindent}{0pt}



\begin{document}
	\thispagestyle{empty}
	
	% ===== Encabezado con logos + texto =====
	\begin{minipage}[c]{0.18\textwidth}
		\centering
		% Cambia por tu logo izquierdo
		\includegraphics[width=0.95\linewidth]{img/logo_usac.jpeg}
	\end{minipage}
	\hfill
	\begin{minipage}[c]{0.60\textwidth}
		\small
		Universidad de San Carlos de Guatemala\\
		Escuela de Ciencias Físicas y Matemáticas\\
		Nombre estudiante: Carlos Pablo\\
		Carnet: 202604171 \\
		Programación 1\\
	\end{minipage}
	\hfill
	\begin{minipage}[c]{0.18\textwidth}
		\centering
		% Cambia por tu logo derecho
		\includegraphics[width=1.4\linewidth]{img/logo_ecfm.jpg}
	\end{minipage}
	
	\vspace{0.5cm}
	
	% Línea horizontal superior (gruesa)
	\noindent\rule{\textwidth}{1.2pt}
	
	\vspace{0.2cm}
	
	% ===== Título =====
	\begin{center}
		{\Large\scshape Programación en la Física}\\[0.3em]
	\end{center}
	
	\vspace{0.1cm}
	
	% Fecha
	\begin{center}
		\small\scshape 3 de febrero de 2026
	\end{center}
	
	\vspace{0.2cm}
	
	% Línea horizontal inferior (gruesa)
	\noindent\rule{\textwidth}{1.2pt}
	
	\vspace{0.6cm}
	
	% ===== Caja de resumen =====
	\noindent
	\colorbox{gray!35}{%
		\parbox{\textwidth}{%
			\vspace{0.6em}
			\textbf{Resumen}\\[0.3em]
			\small
			La programación es fundamental para la ciencia moderna, y la astrofísica no
			es una excepción. Esta permite estudiar y modelar fenómenos del 
			universo que no pueden recrearse directamente. A través de 
			simulaciones computacionales, es posible analizar el movimiento 
			de cuerpos celestes, la evolución de las estrellas, la formación 
			de galaxias y la expansión del universo. Además, facilita el 
			procesamiento de grandes cantidades de datos obtenidos por 
			telescopios y satélites, lo que ayuda a descubrir planetas, 
			estudiar estrellas y observar fenómenos extremos como agujeros 
			negros. También permite crear visualizaciones que facilitan la 
			comprensión y divulgación del conocimiento astronómico.
			\vspace{0.8em}
		}%
	}
	
	\section{Introducción}
	La programación es una herramienta esencial para el 
	desarrollo y avance de la física moderna. A lo largo de la historia, la 
	física ha buscado explicar los fenómenos naturales mediante teorías, 
	leyes y modelos matemáticos. Sin embargo, con el crecimiento de la 
	complejidad de los problemas científicos, la programación ha permitido 
	realizar cálculos, simulaciones y análisis de datos que serían 
	prácticamente imposibles de resolver únicamente de forma manual. Por esta 
	razón, la programación no solo facilita el trabajo de los físicos, sino 
	que también abre nuevas posibilidades de investigación y descubrimiento.
	Ahora mismo, la rama de la física que más interés me produce es la astrofísica, 
	así que en este documento se explicará de qué manera se puede utilizar la programación
	para la investigación en este campo de estudio.

	\section{Desarrollo}
	La programación es una herramienta clave en la astrofísica, 
	ya que permite estudiar fenómenos que ocurren a distancias lejanas, en 
	escalas de tiempo muy largas o en condiciones imposibles de recrear en 
	la Tierra. A través de modelos computacionales, es posible simular el 
	comportamiento del universo y analizar datos obtenidos por telescopios 
	y satélites. \\

	Uno de los usos más importantes es la simulación del movimiento de cuerpos 
	celestes. En astrofísica, se estudia cómo interactúan gravitacionalmente 
	planetas, estrellas, asteroides y galaxias. Mediante programación, se 
	pueden crear modelos basados en las leyes de Newton o en la relatividad 
	general para predecir órbitas, colisiones o la formación de sistemas 
	planetarios. Estas simulaciones permiten comprender mejor la evolución 
	del universo y anticipar eventos astronómicos. \\

	También se utiliza la programación para simular la evolución de las 
	estrellas. Las estrellas pasan por diferentes etapas a lo largo de 
	millones o miles de millones de años. Con programas especializados, 
	es posible modelar procesos como la fusión nuclear, el colapso 
	gravitacional o la explosión de supernovas. Esto ayuda a entender 
	cómo se forman los elementos químicos y cómo nacen y mueren las estrellas.\\

	Otro campo importante es el estudio de galaxias y cosmología. Se utilizan 
	simulaciones computacionales para analizar la formación de galaxias, 
	la distribución de la materia oscura y la expansión del universo. 
	Estos modelos permiten comparar predicciones teóricas con observaciones 
	reales obtenidas por telescopios espaciales. \\

	La programación también es esencial para el procesamiento y análisis de 
	datos astronómicos. Los telescopios modernos generan enormes cantidades 
	de información en forma de imágenes, espectros y mediciones. Con ayuda de 
	algoritmos y técnicas de análisis de datos, los científicos pueden 
	identificar planetas fuera del sistema solar, estudiar la composición de 
	las estrellas o detectar señales provenientes de fenómenos extremos como 
	agujeros negros o púlsares. \\

	Además, la programación facilita la creación de visualizaciones científicas.
	 Estas representaciones gráficas permiten observar simulaciones del universo,
	  lo que ayuda tanto a la investigación como a la educación y divulgación científica.

	\section{Conclusiones}
	La programación es una herramienta imprescindible para la investigación 
	científica actual. Enfocándose en la astrofísica, la programación permite 
	modelar fenómenos astrofísicos mediante simulaciones, analizar infinidad 
	de datos y visualizar los resultados de investigaciones. Gracias a estas 
	herramientas, es posible un estudio del universo con mayor precisión, 
	comprendiendo procesos complejos y realizando descubrimientos que amplían 
	el conocimiento que tenemos sobre el cosmos.

\end{document}

\end{document}
